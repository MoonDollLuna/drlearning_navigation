\chapter{Ficheros de configuración}

Este anexo contiene los ficheros de configuración desarrollados durante el trabajo. En total, se han desarrollado un total de \textbf{12} ficheros de configuración, aunque se pueden dividir en \textbf{4} tipos básicos:

\begin{itemize}
	\item \textbf{Fichero base:} Este fichero contiene la configuración básica y sirve de base para el resto de ficheros (siendo importados por éstos)
	\item \textbf{Ficheros de entrenamiento:} Este fichero contiene la configuración necesaria para el entrenamiento de los agentes, incluyendo los parámetros usados durante \textit{Deep Q-Learning}. Hay \textbf{8} variantes de este fichero dependiendo de la combinación de variantes a utilizar.
	\item \textbf{Ficheros de \textit{benchmark}:} Este fichero contiene la configuración usada para la evaluación de los agentes. Hay \textbf{2} variantes de este fichero, dependiendo del conjunto de datos a usar (\textit{Matterport3D} o \textit{Gibson}).
	\item \textbf{Fichero de generación de video:} Este fichero contiene la configuración usada para generar video del agente durante un episodio. Contiene un bloque con una clave \textit{dummy} de \textit{RL} para poder usar las utilidades de generación de video de \textit{Habitat}.
\end{itemize}

Los ficheros creados (junto a su documentación en inglés) se muestran a continuación.

\section{Fichero base}

\begin{lstlisting}[language=yaml]
# REACTIVE NAVIGATION
# General evaluation / showcase configuration
# Developed by Luna Jimenez Fernandez
#
# This file contains all general parameters to be used while evaluating 
# or showcasing the trained agents.
# 
# In addition, the contents of this file are also used as a basis for the configuration
# used while training the agents. Training specific parameters can be found in the 
# method-specific files (such as reactive_pointnav_train.yaml)
#
# Finally, some of the parameters are specified via arguments during program launch:
#	- Agent type
#	- Dataset and splits to be used
#	- Path to pre-trained weights
#

ENVIRONMENT:
  MAX_EPISODE_STEPS: 1000

SIMULATOR:
  AGENT_0:
    SENSORS: ['RGB_SENSOR', 'DEPTH_SENSOR']
  RGB_SENSOR:
    WIDTH: 256
    HEIGHT: 256
  DEPTH_SENSOR:
    WIDTH: 256
    HEIGHT: 256
    
  # Physics are explicitely disabled, to avoid segmentation faults
  HABITAT_SIM_V0:
    ENABLE_PHYSICS: False
    # PHYSICS_CONFIG_FILE: "None"
    

# DATASET: Data path, split and name are specified via argument
DATASET:
  TYPE: PointNav-v1
#  SPLIT: 
#  DATA_PATH:
#  NAME:

TASK:
  TYPE: Nav-v0
  SUCCESS_DISTANCE: 0.3
  SENSORS: ['POINTGOAL_WITH_GPS_COMPASS_SENSOR']
  POSSIBLE_ACTIONS: ['STOP', 'MOVE_FORWARD', 'TURN_LEFT', 'TURN_RIGHT']
  POINTGOAL_WITH_GPS_COMPASS_SENSOR:
    GOAL_FORMAT: "POLAR"
    DIMENSIONALITY: 2
  GOAL_SENSOR_UUID: pointgoal_with_gps_compass
  MEASUREMENTS: ['DISTANCE_TO_GOAL', 'SUCCESS', 'SPL', 'SOFT_SPL', 'COLLISIONS']
  SUCCESS:
    SUCCESS_DISTANCE: 0.3
\end{lstlisting}

\section{Ficheros de entrenamiento}

\begin{lstlisting}[language=yaml]
# REACTIVE NAVIGATION
# Reactive Navigation Agent training configuration
# Developed by Luna Jimenez Fernandez
#
# This file contains all the specific parameters used by the agent training, including
# both the Deep Q Learning and the Reward computation parameters
#
# Note that this config is added on top of "base_config.yaml", so both files need to be configured
# The following arguments can be found in "base_config.yaml":
#	- Steps per episode (ENVIRONMENT->MAX_EPISODE_STEPS)
#	- Goal radius (TASK->SUCCESS_DISTANCE)
#
# NOTE: Not all parameters are configured via config, the following parameters can be specified
# as arguments when launching the script:
#	- Agent type
#	- Dataset and splits to be used
#	- Path to pre-trained weights
#

VERBOSE: False

BASE_TASK_CONFIG_PATH: "configs/base_config.yaml"
TRAINER_NAME: "reactive"
ENV_NAME: "ReactiveNavEnv"
SIMULATOR_GPU_ID: 0
TORCH_GPU_ID: 0
VIDEO_OPTION: []
# Can be uncommented to generate videos during training.
# VIDEO_OPTION: ["disk", "tensorboard"]
TENSORBOARD_DIR: "Evaluation/Reactive/Tensorboard"
VIDEO_DIR: "Evaluation/Reactive/Video"
# Evaluate on all episodes
TEST_EPISODE_COUNT: -1

SENSORS: ['DEPTH_SENSOR', 'POINTGOAL_WITH_GPS_COMPASS_SENSOR']

CHECKPOINT_FOLDER: "Training/Reactive/Checkpoints"
TRAINING_LOG_FOLDER: "Training/Reactive/Log"
# If True, the log will not output messages to the console screen during training
LOG_SILENT: False
EVAL_CKPT_PATH_DIR: "Training/Reactive/Checkpoints"

# One of these two parameters must be present:
#	TOTAL_NUM_STEPS: Maximum number of steps the agent will take across all epochs
#	NUM_UPDATES: Number of updates (completed episodes) that have been performed
# Training stops when either of these values are reached

# TOTAL_NUM_STEPS: 2000.0
NUM_UPDATES: 15000
LOG_INTERVAL: 25
NUM_CHECKPOINTS: 100

# Reinforcement Learning specific configs
RL:
  # Seed used for all experiments. Can be commented to use a random seed
  seed: 0

  # Deep Q-Learning parameters
  DQL:
    # Learning rate of the neural network
    learning_rate: 0.001
    # Maximum size of the Experience Replay (once full, older experiences will be removed)
    er_size: 20000
    # Batch size when sampling the Experience Replay
    batch_size: 64
    # Batches of experiences are split into chunks of this size during training
    # Reduces training speed, but improves memory usage
    training_batch_size: 32
    # Gamma value (learning rate of DQL)
    gamma: 0.99
    # Epsilon value (initial chance to perform a random action due to exploration-exploitation)
    epsilon: 1.00
    # Minimum epsilon value, achieved after a percentage of epochs (min_epsilon_percentage)
    min_epsilon: 0.05
    # Percentage of epochs (between 0 and 1) after which epsilon will reach min_epsilon.
    # The value of epsilon will decrease linearly from epsilon to min_epsilon
    min_epsilon_percentage: 0.8
    # Chooses between standard DQL (False) or Prioritized DQL (True)
    prioritized: False
    # (PRIORITIZED ONLY) Alpha value (priority degree). The higher alpha is, the higher the probability of choosing higher error experiences is
    prioritized_alpha: 0.5
    # (PRIORITIZED ONLY) Beta value (bias degree). The higher the value is, the less weight variations have (to avoid big oscillations)
    prioritized_beta: 0.5
  
  # Image pre-processing parameters (used to compute the rewards)
  IMAGE:
    # Pixels to be trimmed from both bottom and top of the image (the depth view seen by the camera)
    # This parameter is relevant since the robot is an embodied agent that will always see the floor (and, in the case of houses,
    # the roof) at a constant height
    # Therefore, it can be trimmed without problem
    trim: 35
    # Threshold to consider a part of the image an obstacle. Note that the image is a grayscale image from 0.0 to 1.0, where 0 (black) means the closest and 1.0 (white) means the furthest
    # This also doubles as the maximum distance to an obstacle.
    obstacle_threshold: 0.15
    # Minimum area (in pixels) for contours / columns. Contours / columns smaller than this size will be ignored
    min_contour_area: 250
    # (COLUMN REWARDS ONLY) Total columns to be used when using the column reward method. Ignored when using the contour reward_method
    reward_columns: 8
    
  # Reward parameters
  REWARD:
    # Reward method to be used. There are two possibilities:
    #	- contour: Contour based approach, imitating the original laser-based proposal
    #	- column: Column based approach, dividing the image into smaller columns and computing each column as obstacle / no obstacle.
    reward_method: contour
    # Approximate distance (in simulator units) at which obstacles are when they are at the threshold.
    # Can also be understood as the maximum distance the camera will detect obstacles
    obstacle_distance: 2
    # Positive gain applied to the attractive field, to increase its weight
    attraction_gain: 100
    # Positive gain applied to the repulsive field, to increase its weight
    repulsive_gain: 15
    # Value used to limit the repulsive field's maximum value
    repulsive_limit: 0.04
    # Percentage (between 0 and 1). When the goal is closer than repulsive_goal_influence * obstacle_distance, the effect of the repulsive field gets decreased
    repulsive_goal_influence: 0.75
    # Success reward. Note that positive rewards will also be clipped to this value
    success_reward: 10
    # Slack penalty, added to the reward each non-final episode to ensure that the agent doesn't end in a loop of doing actions without reward to avoid a penalty
    slack_penalty: -0.25
    # Failure penalty. Note that negative rewards will also be clipped to this value
    failure_penalty: -100
    # If True, the episode will end immediately if a collision is detected
    collisions: False
\end{lstlisting}

\section{Ficheros de \textit{benchmark}}

\begin{lstlisting}[language=yaml]
# REACTIVE NAVIGATION
# Benchmark configuration
# Developed by Luna Jimenez Fernandez
#
# This file contains all general parameters to be used while benchmarking
# the agents
#
# In essence, this is a variation of the base file to be used
# when evaluating the agents

# BENCHMARKING OPTIONS
VIDEO_OPTION: ["disk"]
VIDEO_DIR: "Video/Reactive"
SEED: 0

ENVIRONMENT:
  MAX_EPISODE_STEPS: 500

SIMULATOR:
  AGENT_0:
    SENSORS: ['RGB_SENSOR', 'DEPTH_SENSOR']
  RGB_SENSOR:
    WIDTH: 256
    HEIGHT: 256
  DEPTH_SENSOR:
    WIDTH: 256
    HEIGHT: 256
    
  # Physics are explicitely disabled, to avoid segmentation faults
  HABITAT_SIM_V0:
    ENABLE_PHYSICS: False
    
# DATASET: Dataset path, name and split to be used must be specified here
DATASET:
  TYPE: PointNav-v1
  SPLIT: "val"
  DATA_PATH: "./data/datasets/pointnav/gibson/v1/{split}/{split}.json.gz"
  NAME: "gibson"

TASK:
  TYPE: Nav-v0
  SUCCESS_DISTANCE: 0.3
  SENSORS: ['POINTGOAL_WITH_GPS_COMPASS_SENSOR']
  POSSIBLE_ACTIONS: ['STOP', 'MOVE_FORWARD', 'TURN_LEFT', 'TURN_RIGHT']
  POINTGOAL_WITH_GPS_COMPASS_SENSOR:
    GOAL_FORMAT: "POLAR"
    DIMENSIONALITY: 2
  GOAL_SENSOR_UUID: pointgoal_with_gps_compass
  MEASUREMENTS: ['DISTANCE_TO_GOAL', 'SUCCESS', 'SPL', 'SOFT_SPL', "COLLISIONS"]
  SUCCESS:
    SUCCESS_DISTANCE: 0.3

# Dummy RL config, to allow usage with video
RL:
  dummy: True
\end{lstlisting}

\section{Fichero de generación de video}

\begin{lstlisting}[language=yaml]
# REACTIVE NAVIGATION
# Video
# Developed by Luna Jimenez Fernandez
#
# This file contains all general parameters to be used 
# in order to generate video of the agents
#

BASE_TASK_CONFIG_PATH: "configs/base_config.yaml"
VIDEO_OPTION: ["disk"]
VIDEO_DIR: "Video"
SEED: 0

# Dataset is still specified via console

# Add a dummy RL section, to avoid errors
# (Habitat Lab expects video to only be created for RL agents)
RL:
 dummy: True

\end{lstlisting}

