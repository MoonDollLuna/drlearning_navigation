\chapter{Introducción y objetivos}
%%---------------------------------------------------------

En este capítulo se realizará una breve introducción a los contenidos que se desarrollarán a lo largo de la memoria. Tras ésta, se presentará la motivación que ha propiciado el desarrollo de este trabajo. Finalmente, se describirá la estructura que seguirá la memoria.

\section{Introducción}

\section{Motivación}

\section{Estructura}

Esta memoria está dividida en un total de 7 capítulos, que serán descritos brevemente a continuación:

\begin{itemize}
	\item \textbf{Capítulo 1:} En este capítulo se introduce el tema desarrollado , la motivación y la estructura de la memoria.
	\item \textbf{Capítulo 2:} En este capítulo se describe el problema a resolver, detallando los antecedentes al trabajo realizado e introduciendo los objetivos a alcanzar.
	\item \textbf{Capítulo 3:} En este capítulo se realiza un estudio del estado del arte de los campos relacionados con el trabajo: aprendizaje por refuerzo (analizando tanto técnicas clásicas como las técnicas modernas utilizando \textit{deep learning}) y algoritmos de navegación automática.
	\item \textbf{Capítulo 4:} En este capítulo se describe en detalle el sistema desarrollado, remarcando las partes que componen los agentes a desarrollar y las diferencias que existen entre ellos.
	\item \textbf{Capítulo 5:} En este capítulo se describe la implementación del sistema, haciendo hincapié tanto en los agentes desarrollados como en la instalación y uso de las herramientas y librerias utilizadas.
	\item \textbf{Capítulo 6:} En este capítulo se detalla la experimentación realizada sobre el sistema, detallando las variables y los experimentos a realizar sobre el sistema. Además, se realiza un análisis de los resultados de los experimentos tanto durante el entrenamiento como en problemas reales.
	\item \textbf{Capítulo 7:} Finalmente, en este capítulo se presentan las conclusiones alcanzadas tras el desarrollo del trabajo, proponiendo posibles lineas de trabajo futuro para continuarlo.
\end{itemize}

Además, se incluye una bibliografía en la que se encuentra la lista de fuentes y referencias usadas a lo largo de la memoria.

%%---------------------------------------------------------