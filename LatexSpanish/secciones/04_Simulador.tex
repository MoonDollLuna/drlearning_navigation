\chapter{Simulador: \textit{Habitat Sim} y \textit{Habitat Lab}}

\section{\textit{Habitat Sim} y \textit{Habitat Lab}}
[UNA INTRODUCCION MAS BREVE AL SIMULADOR COMO TAL]

\subsection{\textit{Habitat Sim}}
[HABLA POR SEPARADO DE HABITAT 1 Y HABITAT 2]

\subsection{\textit{Habitat Lab}}
[HABLA DE HABITAT LAB Y HABITAT BASELINES]

\subsection{Principales conceptos de Habitat Lab}
[DE CADA CONCEPTO, PROBABLEMENTE EXPLICAR QUE ES, EXPLICAR QUE OFRECE POR DEFECTO HABITAT / BASELINES]

[NO SE SI SERÁ NECESARIO HABLAR DEL PROPIO AGENTE AQUÍ O EN UN ANEXO]

[EL ORDEN ES TENTATIVO]

\subsubsection{Entornos}

[ENV, RLENV, NAVRLENV Y LOS METODOS QUE HAY QUE IMPLEMENTAR]

\subsubsection{Tareas}
[PROBABLEMENTE MENCIONAR LAS PRINCIPALES TAREAS OFRECIDAS POR ENCIMA]

\subsubsection{Conjuntos de datos}
[FOTOS DE CADA CONJUNTO DE DATOS]
[MENCIONAR LOS 4 DATASETS DISPONIBLES, Y COMO USARLOS (ESTRUCTURA)]

\subsubsection{Episodios}

\subsubsection{Sensores}

\subsubsection{Ficheros de configuración}
[PROBABLEMENTE MENCIONAR SECCIONES CLAVES Y ELEMENTOS CLAVES QUE SIGNIFICAN]

\subsubsection{Entrenadores}

[AQUI UN PSEUDOCODIGO DEL METODO DE ENTRENAR QUEDARÍA DE LUJO]

\subsubsection{Agentes}

\subsubsection{\textit{Benchmarks}}

\section{Instalación del simulador}

\subsection{Requisitos y versiones}
[INDICAR LAS VERSIONES USADAS DE TODO, Y ESPECIFICAR QUE NO SE GARANTIZA QUE FUNCIONE CON TODO]

\subsection{Proceso de instalación}
[INDICAR TAMBIEN CUDA]