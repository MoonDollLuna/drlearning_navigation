\chapter{Introducción}

En este capítulo se realizará una breve introducción a los contenidos que serán expuestos posteriormente a lo largo de la memoria. Tras ésta presentación, se expondrá la motivación que ha propiciado el desarrollo de este trabajo. Finalmente, se describirá la estructura seguida por la memoria.

\section{Introducción}
LO TIPICO DE INTRO

\section{Motivación}
LO TIPICO DE MOTIVACION

\section{Estructura}


Esta memoria está dividida en un total de 7 capítulos, que serán descritos brevemente a continuación.

\begin{itemize}
	\item \textbf{Capítulo 1:} En este capítulo se introduce el trabajo desarrollado, la motivación que ha llevado a éste y la estructura general de la memoria.
	\item \textbf{Capítulo 2:} En este capítulo se describe en profundidad el problema a resolver, presentando los antecedentes previos al trabajo realizado y detallando los objetivos que se esperan cumplir.
	\item \textbf{Capítulo 3:} En este capítulo se realiza una revisión de las principales técnicas en los campos relacionados con el trabajo: \textit{deep learning} y redes neuronales convolucionales, aprendizaje por refuerzo (estudiando tanto las técnicas clásicas como las técnicas de aprendizaje por refuerzo profundo) y algunos de los principales algoritmos de navegación automática.
	\item \textbf{Capítulo 4:} En este capítulo se presentan tanto \textit{Habitat Sim} como \textit{Habitat Lab}, las principales herramientas usadas durante el desarrollo del trabajo. De estas herramientas se comenta además su instalación y sus dependencias. Finalmente, se exponen los principales conceptos de Habitat Lab, explicando su funcionamiento y dando detalles de la implementación propia realizada.
	\item \textbf{Capítulo 5:} En este capítulo se detalla el diseño del agente de navegación reactiva propuesto. Se describe tanto la representación del conocimiento (estado, acciones y recompensas) como la arquitectura, el método de actuación y el entrenamiento llevado a cabo por el agente. Finalmente, se realiza una breve explicación del funcionamiento y la arquitectura del resto de agentes usados como \textit{benchmarks} y comparativas ofrecidos por \textit{Habitat Lab}.
	\item \textbf{Capítulo 6:} En este capítulo se detalla la experimentación realizada, indicando los parametros utilizados. Además, se presentan los resultados y el rendimiento obtenido por los agentes tanto durante el entrenamiento como durante la evaluación posterior.
	\item \textbf{Capítulo 7:} Finalmente, en este capítulo se presentan las conclusiones alcanzadas tras el desarrollo del trabajo, proponiendo posibles lineas de trabajo futuro para continuarlo.
\end{itemize}

Además, al final de la memoria se incluye una bibliografía en la que se encuentra la lista de fuentes y referencias usadas a lo largo de ésta.

%%---------------------------------------------------------